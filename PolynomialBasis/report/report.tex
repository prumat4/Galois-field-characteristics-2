\documentclass{article}
\usepackage[english,ukrainian]{babel}
\usepackage[letterpaper,top=2cm,bottom=2cm,left=3cm,right=3cm,marginparwidth=1.75cm]{geometry}
\usepackage{amsmath}
\usepackage{graphicx}
\usepackage{caption}
\usepackage{color}
\usepackage[colorlinks=true, allcolors=blue]{hyperref}

\begin{document}

\newpage
\section{Реалізація}
\quad Клас GaloisField представляє елементи поля Галуа зі степенем 239. Він включає в себе арифметичні операції, такі як додавання, множення, піднесення до степеня, знаходження обернених чисел та сліду. Клас також підтримує ініціалізацію з бітового набору, шістнадцяткового рядка та випадкову генерацію бітів.

\section{Середній час роботи операцій}
\quad
\begin{table}[h]
\centering
\begin{tabular}{|c|c|c|c|c|}
\hline
\textbf{Розмір} & \textbf{Додавання (ns)} & \textbf{Множення (ns)} & \textbf{Піднесення до степеня (ms)} & \textbf{Обернення (ms)} \\
\hline
16 & 128.57 & 45312.8 & 178.07 & 352.46 \\
32 & 126.45 & 79633.6 & 191.5 & 357.66 \\
64 & 116.43 & 155148 & 196.44 & 349.45 \\
128 & 123.15 & 333578 & 226.35 & 356.46 \\
239 & 148.22 & 792021 & 285.47 & 379.63 \\
\hline
\end{tabular}
\caption{Середній час виконання операцій}
\label{tab:comparison}
\end{table}

\quad \textcolor{red}{\textbf{!!!}} В таблиці фігурує змінна \textit{'розмір'}, для неї я генерував випадкові бітові послідновності, тобто, якщо розмір 16, то це означає, що потенційно лише перші 16 бітів послідовності можуть бути 1, а всі інші 0.

\section{Тестування класу GaloisField}

Ці тести перевіряють функціональність операцій у класі GaloisField.

\subsection{Addition Tests}
\begin{itemize}
    \item Addition Test:
        \begin{itemize}
            \item вхід: 
                \begin{itemize}
                    \item $a$: 4fc1c8f723d908a8459557b73b1d9335db15946ae1cd3c638b407ea812c
                    \item $b$: 6c6e49eca4f2b4b5684cc678e5b1449500a6ae03b4e732eaacdec175ccd
                \end{itemize}
            \item  результат: 23af811b872bbc1d2dd991cfdeacd7a0dbb33a69552a0e89279ebfddde1
        \end{itemize}
    \item Neutral Element Test:
        \begin{itemize}
            \item вхід: 
                \begin{itemize}
                    \item $a$: 4fc1c8f723d908a8459557b73b1d9335db15946ae1cd3c638b407ea812c
                    \item $zero$: 0000000000000000000000000000000000000000000000000000000000000
                \end{itemize}
            \item  результат: 4fc1c8f723d908a8459557b73b1d9335db15946ae1cd3c638b407ea812c
        \end{itemize}
\end{itemize}
\subsection{Multiplication Tests}
\begin{itemize}
    \item Multiplication Test:
        \begin{itemize}
            \item вхід: 
                \begin{itemize}
                    \item $a$: 4fc1c8f723d908a8459557b73b1d9335db15946ae1cd3c638b407ea812c
                    \item $b$: 6c6e49eca4f2b4b5684cc678e5b1449500a6ae03b4e732eaacdec175ccd
                \end{itemize}
            \item результат: 28f0923c10c934e09527bfa23cd7e2b853b3b2f3e0b3ec7aea2152073287
        \end{itemize}
    \item Neutral Element Test:
        \begin{itemize}
            \item вхід: 
                \begin{itemize}
                    \item $a$: 4fc1c8f723d908a8459557b73b1d9335db15946ae1cd3c638b407ea812c
                    \item $one$: 0000000000000000000000000000000000000000000000000000000000001
                \end{itemize}
            \item результат: 4fc1c8f723d908a8459557b73b1d9335db15946ae1cd3c638b407ea812c
        \end{itemize}
    \item To Square Test:
        \begin{itemize}
            \item вхід: 
                \begin{itemize}
                    \item $a$: 4fc1c8f723d908a8459557b73b1d9335db15946ae1cd3c638b407ea812c
                \end{itemize}
            \item результат: 50b6456108378713c37d3039cf1792aec1b5f9dfbebbc0a6c770a75d693e
        \end{itemize}
    \item To Power Of Test:
        \begin{itemize}
            \item вхід: 
                \begin{itemize}
                    \item $a$: 4fc1c8f723d908a8459557b73b1d9335db15946ae1cd3c638b407ea812c
                    \item $b$: 6c6e49eca4f2b4b5684cc678e5b1449500a6ae03b4e732eaacdec175ccd
                \end{itemize}
            \item результат: 17c762a0c47b27abcd6d274ad5b5c738ff5ac768d320f24514b1f483ddaa
        \end{itemize}
\end{itemize}
\subsection{Utilities Tests}
\begin{itemize}
    \item Trace Test:
        \begin{itemize}
            \item вхід: 
                \begin{itemize}
                    \item $a$: 4fc1c8f723d908a8459557b73b1d9335db15946ae1cd3c638b407ea812c
                    \item $b$: 6c6e49eca4f2b4b5684cc678e5b1449500a6ae03b4e732eaacdec175ccd
                \end{itemize}
            \item  результат: для 'a' = 0, для 'b' = 1
        \end{itemize}
    \item Inverse Test:
        \begin{itemize}
            \item вхід: 
                \begin{itemize}
                    \item $a$: 4fc1c8f723d908a8459557b73b1d9335db15946ae1cd3c638b407ea812c
                    \item $b$: 6c6e49eca4f2b4b5684cc678e5b1449500a6ae03b4e732eaacdec175ccd
                \end{itemize}
            \item результат для 'a': 5832c6cdadb2067298e6c340ce3eadf6810ed043badeb297a8219eebc277
            \item результат для 'b': 3d4b7cbaf4dec964719450d456eb5c6074b00df53fa4e11d19b248328389
        \end{itemize}
\end{itemize}

\subsection{Complex Tests}
\begin{itemize}
    \item Distributivity Test:
        \begin{itemize}
            \item вхід: 
                \begin{itemize}
                    \item $a$: 4fc1c8f723d908a8459557b73b1d9335db15946ae1cd3c638b407ea812c
                    \item $b$: 6c6e49eca4f2b4b5684cc678e5b1449500a6ae03b4e732eaacdec175ccd
                    \item $c$: 09d7f58ff5398570a5ba840d9f0fc5c806f5353788a4c0b8488e4e62d2a
                \end{itemize}
            \item результат: True
        \end{itemize}
    \item Neutral Test:
        \begin{itemize}
            \item вхід: 
                \begin{itemize}
                    \item $a$: 4fc1c8f723d908a8459557b73b1d9335db15946ae1cd3c638b407ea812c
                \end{itemize}
            \item результат: 1
        \end{itemize}
\end{itemize}

\bibliographystyle{alpha}
\end{document}